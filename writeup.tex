\title{Bomb Write Up}
\author{
        Elijah Harmon
}
\date{\today}

\documentclass[12pt]{article}

\begin{document}
\maketitle

\section{Phase 1}
\paragraph{Answer}
swordfish

\paragraph{Discovery}
I ran strings on bomb and found the string swordfish by a bunch of other strings that outputed to the screen. So I typed it in and it worked.

\paragraph{Thoughts}
I am happy that phase 1 was easy to solve. I hope the rest of the phases are just as easy.

\section{Phase 2}
\paragraph{Answer}
eharmon (Current user's username)

\paragraph{Patch}
The patch can be found in the patch folder and is called bomb-phase2. I patched line 0x401466 in \_start. I beleave this was code to stop other programs from running bomb. It now can be debugged with GDB but if you run it normally it will say it 'segment fault'ed. It doesn't actually seg fault. Probably because now it has to be run in something else to work properly. Might chage it later to a JMP instead of a JE.

\begin{verbatim}
ORIGINAL
401466:   75 60                   jne    4014c8 <_start+0x93>

PATCHED
401466:   74 60                   je    4014c8 <_start+0x93>
\end{verbatim}

\paragraph{Discovery}
Either I was doing extra or this phase's difficulty exponetally rose. I had to patch the bomb file to be able to run it in GDB. If the file is ran now, it will output that is segment faulted but it didn't. The program is a liar. Now I could run it in GDB and it said it couldn't find a main(). So I had it start at \_start. I steped through until I got to phase 1. I entered the answer for phase 1 when I got to it's puts(). I continued to step through until I got to phase 2. For it's puts() I just entered 'J'. I steped until I went past the $<$getenvs@plt$>$, after that I noticed a lot of registers changed. I looked rax and noticed it had my username in it. So im assuming getenvs gets the enviroment variables (aka my username).

\paragraph{Thoughts}
This phase was a lot harded than phase 2 and now I am scared of the rest of the phases.

\section{Phase 3}
\paragraph{Answer}
12345 (Can have multiple answers)

\paragraph{Patch}
This patched version changes the JE from last patch to a JMP so it will always run.

\begin{verbatim}
PREV PATCH
401466:   74 60                   je    4014c8 <_start+0x93>

PATCHED
401466:   eb 60                   jmp    4014c8 <_start+0x93>
\end{verbatim}

\paragraph{Discovery}
Math is no fun. So I ran through the program once and I learned that this phase would catinate your previous entry (phase 2) to your entry from this phase. It would then take the length of the new string and store it (lets call it var a). Then it would run a sscan on the catinated string looking for '\%u' numbers. It takes the number it found in the string and stores it into the stack (var b). Then it divides var b by var a. Stores that result into var b. Shifts it to the right by 0x2 and stores it into var b. then multiplies var b against 0xaaaaaaab. Then the higher 32 bits are moved into EAX. EAX is then shifted to the right by 1 bit and stored into var b. 
Var a and var b are then compared. 
So the unsigned int will be divided by the length of the string, then divided by 4. Then multiplied by 0xaaaaaaab. Then divided by 32. Then divived by 2.

\paragraph{Thoughts}
This phase was hard but not as big of a difficulty jump compared to the last one. I also don't like math so...

\end{document}